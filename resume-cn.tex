\documentclass{resume} % Use the custom resume.cls style

\usepackage{ctex}
\usepackage[left=0.4 in,top=0.8in,right=0.4 in,bottom=0.4in]{geometry}

\newcommand{\tab}[1]{\hspace{.2667\textwidth}\rlap{#1}}
\newcommand{\itab}[1]{\hspace{0em}\rlap{#1}}

\name{Chen Su}
\address{\href{mailto:ghosind@gmail.com}{ghosind@gmail.com} \\ \href{https://ghosind.com}{ghosind.com} \\ \href{https://github.com/ghosind}{github.com/ghosind}}

\begin{document}

%----------------------------------------------------------------------------------------
%----------------------------------------------------------------------------------------
%	EDUCATION SECTION
%----------------------------------------------------------------------------------------

\begin{rSection}{教育背景}

  {\bf 工学学士} 计算机科学与技术 \hfill {2017.6}

\end{rSection}

%----------------------------------------------------------------------------------------
% TECHNICAL STRENGTHS
%----------------------------------------------------------------------------------------
\begin{rSection}{个人技能}
  \begin{itemize}
    \itemsep -3pt {}
    \item 熟悉Golang、JavaScript/TypeScript、C等编程语言,具有相应的的工作或项目经验。
    \item 具有分布式后端服务架构设计经验,有能力独立或参与团队合作进行架构设计、技术选型及开发。
    \item 熟悉常用的MySQL、MongoDB、Redis等数据库系统,对ElasticSearch有一定了解。
    \item 熟悉Docker使用及Dockerfile的编写,对Kubernetes有一定了解。
    \item 熟悉常用的HTTP、gRPC等通信方式,对GraphQL、SOAP等协议有所了解。
    \item 熟悉AWS云服务,具有多年基于AWS云服务的应用程序开发经验,对阿里云及微软Azure有一定了解。
    \item 对Linux系统有一定的了解,熟悉Linux下基本命令的使用以及Shell脚本的编写。
    \item 了解消息队列系统机制,有RabbitMQ与ActiveMQ使用经验。
    \item 具有GitHub Actions、GitLab CI/CD等自动化工具使用经验,熟悉通过Shell脚本构建自动化工具。
    \item 具有国际化团队合作经验(北美、欧洲、东南亚),有一定的英文阅读表达能力。
    \item 参与Golang、Deno等多个大型开源项目,并作为导师指导CNCF LFX Mentorship的KubeEdge课题。
  \end{itemize}
\end{rSection}

%----------------------------------------------------------------------------------------
% WORK EXPERIENCE
%----------------------------------------------------------------------------------------
\begin{rSection}{工作经验}

\textbf{FatCoupon} \hfill 2019.5 - 至今\\
技术负责人、后端工程师 \hfill \textit{}
\begin{itemize}
  \itemsep -3pt {}
  \item 负责技术团队建设及管理工作,主要负责后端技术团队日常管理工作事务,及前端项目核心模块的设计。
  \item 作为主要负责人负责多个项目从0到1落地,工作职责包括技术选型、服务端架构设计以及核心功能实现等。
  \item 设计并实现基于AWS Lambda、Docker等基础设施的跨项目微服务架构体系,开发基于Golang、Node.js等语言实现的服务端系统。
  \item 实现并维护内部使用的Web服务、数据库处理、消息处理等基础框架,构建支付网关、动态配置中心、任务队列等基础设施服务,建立通用邮件、消息推送服务等跨项目中间件服务。
  \item 优化基于AWS云服务的内部网络拓扑结构,设计多级缓存逻辑,降低超75\%的内部网络通信成本。
  \item 构建基于多维度身份验证(邮件、短信、TOTP)等方式的服务安全体系,为服务数据提供安全性保障。
  \item 开发内部运维自动化工具,提供服务健康检测汇报、冷数据归档、日志备份处理等功能。
  \item 基于React、AntDesign构建服务端控制平台,实现服务配置管理等基础服务的图形化处理界面。
\end{itemize}

\textbf{Timepop} \hfill 2018.9 - 2019.2\\
后端工程师 \hfill \textit{}
\begin{itemize}
  \itemsep -3pt {}
  \item 作为主要负责人开发并维护基于Node.js、MySQL、GraphQL的社交平台后端服务。
  \item 开发基于FoF等推荐算法的好友推荐系统,定时分析用户数据,为用户提供伪实时二度好友推荐。
  \item 参与基于AWS云服务、Docker的服务端架构设计,基于GitLab CI/CD和Shell脚本实现服务自动化部署。
  \item 参与部分移动端React Native界面及基于Websocket的即时通信功能实现。
\end{itemize}

\textbf{上海店达} \hfill 2018.4 - 2018.8\\
小组负责人、后端工程师 \hfill \textit{}
\begin{itemize}
  \itemsep -3pt {}
  \item 作为负责人带领包括四名后端、一名测试工程师的项目组,并负责与产品经理、线下业务团队对接工作。
  \item 主要负责通过Node.js开发基于Express框架的后端服务,并对内提供Hprose RPC接口。
  \item 重构多个复杂业务逻辑,将多个频发超时接口处理时间稳定至常数级。
  \item 带领小组成员制定代码规范,完成服务依赖版本升级,对部分老服务进行优化重构。
  \item 参与并配合前端团队进行前端页面的实现,并独立实现合同打印等功能。
\end{itemize}

\textbf{杭州品茗} \hfill 2017.5 - 2017.11\\
软件工程师 \hfill \textit{}
\begin{itemize}
  \itemsep -3pt {}
  \item 基于Autodesk Revit、C\#、C++开发三维建筑信息管理系统(BIM)。
  \item 开发基于WPF、WinForm以及MVVM设计模式的Windows桌面程序。
  \item 参与基于2D CAD图纸的3D模型建模功能开发,实现管道建模等功能。
  \item 参与跨团队间基于SQLite实现的BIM项目与CAD数据共享开发。
\end{itemize}

\end{rSection}

%----------------------------------------------------------------------------------------
% PROJECTS
%----------------------------------------------------------------------------------------
\begin{rSection}{项目经历}
  \vspace{-1.25em}

  \item \textbf{FatCoupon} {} \hfill \href{https://fatcoupon.com}{fatcoupon.com}
  \begin{itemize}
    \itemsep -3pt {}
    \item 设计并实现基于Docker、AWS Lambda函数计算构建微服务与Serverless相结合的架构体系,并通过HTTP、gRPC、消息队列以及Lambda调用实现服务间通信。
    \item 基于AWS服务以及消息队列构建消息推送、邮件、短信等基础服务,实现订阅列表、模板渲染等功能,在服务商频率限制下提供日均百万级消息处理能力。
    \item 设计并实现订单获取系统,从多个上游服务获取并分析处理订单数据,实现日均千万级订单数据的处理。
    \item 构建支持支付宝、信用卡、PayPal等国内外主流支付方式的支付网关,为内部服务统一支付接口。
    \item 实现礼品卡基础服务,提供多个上游平台的礼品卡购买、校验以及统一化的服务接口。
    \item 扩展AWS CloudFront CDN功能,提供自定义请求参数控制实现尺寸修改、缓存第三方图片等功能。
  \end{itemize}

  \item \textbf{SnapThePrice} {} \hfill \href{https://snaptheprice.com}{snaptheprice.com}
  \begin{itemize}
    \itemsep -3pt {}
    \item 基于OpenAI GPT4o大语言模型及Google Vision AI,实现图片搜索、准确度分析及关键字优化等功能。
    \item 实现由上游商家服务接口、内部爬虫系统组成数据收集系统,并通过用户请求及定时任务进行调度管理。
  \end{itemize}

  \item \textbf{Kash.ly返利短链接系统} {} \hfill \href{https://kashly.net}{kashly.net}
  \begin{itemize}
    \itemsep -3pt {}
    \item 设计基于LCG、Base62算法与MySQL实现的非线性伪随机短链接服务,实现纳秒级核心算法处理。
    \item 通过AWS Lambda部署短链接跳转服务,提供高并发下自扩容能力,平均接口处理时间低于30ms。
    \item 基于Levenshtein、Needleman–Wunsch等算法实现毫秒级链接识别服务,智能判断支持的平台站点。
    \item 记录并分析短链接数据,为用户及运营团队提供分时点击量统计、终端设备信息分析等数据。
  \end{itemize}

  \item \textbf{DVM} {} \hfill \href{https://github.com/ghosind/dvm}{github.com/ghosind/dvm}
  \begin{itemize}
    \itemsep -3pt {}
    \item 基于ShellScript编写的Linux/MacOS平台下Deno版本管理工具,提供了Deno版本安装、卸载、版本间切换、别名管理以及版本信息错误检测等功能。
    \item 设计基于Git实现的自我版本管理,实现通过命令完成DVM的一键版本升级。
    \item 项目在GitHub、Gitee等平台发布,提供镜像选择功能,并被列为Gitee官方推荐项目。
  \end{itemize}

\end{rSection}

\end{document}
