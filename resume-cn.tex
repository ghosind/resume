\documentclass{resume} % Use the custom resume.cls style

\usepackage{ctex}
\usepackage[left=0.4 in,top=0.4in,right=0.4 in,bottom=0.4in]{geometry}

\newcommand{\tab}[1]{\hspace{.2667\textwidth}\rlap{#1}}
\newcommand{\itab}[1]{\hspace{0em}\rlap{#1}}

\name{Chen Su}
\address{\href{mailto:ghosind@gmail.com}{ghosind@gmail.com} \\ \href{https://ghosind.com}{ghosind.com} \\ \href{github.com/ghosind}{github.com/ghosind}}

\begin{document}

%----------------------------------------------------------------------------------------
%----------------------------------------------------------------------------------------
%	EDUCATION SECTION
%----------------------------------------------------------------------------------------

\begin{rSection}{教育背景}

  {\bf 工学学士} 计算机科学与技术 \hfill {2017.7}

\end{rSection}

%----------------------------------------------------------------------------------------
% TECHNICAL STRENGTHS
%----------------------------------------------------------------------------------------
\begin{rSection}{个人技能}
  \begin{itemize}
    \itemsep -3pt {}
    \item 熟悉C、Golang、JavaScript/TypeScript等编程语言,具有相应的的工作或项目经验。
    \item 熟悉常用的MySQL、MongoDB、Redis等数据库系统,对ElasticSearch有一定了解。
    \item 对Linux系统有一定的了解,熟悉Linux下基本命令的使用以及Shell脚本的编写。
    \item 熟悉Docker使用及Dockerfile的编写,对Kubernetes有一定了解。
    \item 熟悉AWS云服务,具有四年基于AWS云服务的服务端应用程序开发经验,对阿里云及微软Azure有一定了解。
    \item 具有GitHub Actions、GitLab CI/CD等自动化工具使用经验,熟悉通过Shell脚本构建自动化工具。
    \item 具有国际化团队合作经验(北美、欧洲、东南亚),有一定的英文阅读表达能力。
    \item 对开源具有热情,具有多个开源项目的参与经历。
  \end{itemize}
\end{rSection}

%----------------------------------------------------------------------------------------
% WORK EXPERIENCE
%----------------------------------------------------------------------------------------
\begin{rSection}{工作经验}

\textbf{FatCoupon} \hfill 2019.5 - 至今\\
后端工程师/后端负责人 \hfill \textit{}
\begin{itemize}
  \itemsep -3pt {}
  \item 负责服务端基于AWS Lambda、Docker等基础服务设计并构建跨项目架构体系,开发基于Golang、Node.js等语言实现的服务端系统。
  \item 作为主要负责人参与了多个项目从0到1落地,并负责或参与后续的研发与维护。
  \item 编写项目基础框架,构建动态配置中心、任务管道等基础设施服务,并建立通用邮件、推送等跨项目服务中间件服务。
  \item 优化基于AWS云服务的内部网络拓扑结构,设计多级缓存逻辑,降低超75\%的内部网络通信成本(单月成本由2700美元降低至600美元,通信数据传输量由单月190+TB降低至15TB)。
  \item 构建基于设备设别、验证码校验等方式的服务安全体系,防御脚本恶意登录等攻击性行为。
  \item 基于React、AntDesign构建服务端控制平台,实现服务配置管理等基础服务的图形化处理界面。
\end{itemize}

\textbf{Timepop} \hfill 2018.8 - 2019.2\\
后端工程师 \hfill \textit{}
\begin{itemize}
  \itemsep -3pt {}
  \item 作为主要负责人开发并维护基于Node.js、GraphQL的社交平台后端服务。
  \item 开发基于FoF算法的好友推荐系统,为用户提供伪实时二度好友推荐。
  \item 参与基于AWS、Docker的服务端架构设计,基于GitLab CI/CD和Shell脚本实现服务自动化部署。
  \item 参与部分移动端React Native界面及功能的实现。
\end{itemize}

\textbf{上海店达} \hfill 2018.4 - 2018.8\\
后端工程师/小组Leader \hfill \textit{}
\begin{itemize}
  \itemsep -3pt {}
  \item 主要负责通过Node.js开发基于Express框架的后端服务,并编写前端页面及交互脚本。
  \item 作为采购服务组负责人对接产品经理、线下业务团队,确定需求内容及排期。
  \item 优化历史遗留问题,稳定化多个复杂业务查询逻辑,将多个频发超时接口稳定至常数时间内返回。
  \item 带领小组成员制定代码规范,对部分老服务重构优化。
\end{itemize}

\textbf{杭州品茗} \hfill 2017.5 - 2017.11\\
软件工程师 \hfill \textit{}
\begin{itemize}
  \itemsep -3pt {}
  \item 基于AutoDesk Revit、C\#、C++开发作为公司主要产品的HiBIM。
  \item 开发基于WPF、WinForm的Windows桌面程序。
  \item 参与基于2D CAD图纸的3D模型建模功能开发,实现管道建模等功能。
  \item 基于SQLite实现HiBIM与CAD图纸数据的共享。
\end{itemize}

\end{rSection}

%----------------------------------------------------------------------------------------
% PROJECTS
%----------------------------------------------------------------------------------------
\begin{rSection}{项目经历}
  \vspace{-1.25em}
  
  \item \textbf{FatCoupon} {} \hfill \href{www.fatcoupon.com}{fatcoupon.com}
  \begin{itemize}
    \itemsep -3pt {}
    \item 设计并实现基于Docker、AWS Lambda函数计算构建微服务与无服务(Serverless)架构体系,并通过HTTP、gRPC、消息队列以及Lambda调用实现服务间通信。
    \item 基于AWS服务以及消息队列构建消息推送、邮件等基础服务,实现订阅列表、模板渲染以及日均十万级消息的处理。
    \item 通过定时任务从多个上游服务获取并分析处理订单数据,实现对多个项目每日十万级订单数据的检测、创建、更新以及其奖励机制的处理。
    \item 扩展AWS CloudFront CDN功能,实现根据请求参数修改图片、获取第三方图片等功能。
  \end{itemize}
  
  \item \textbf{Kash.ly返利短链接系统} {} \hfill \href{kashly.net}{kashly.net}
  \begin{itemize}
    \itemsep -3pt {}
    \item 设计基于LCG、Base62算法与MySQL实现,提供核心逻辑平均处理时间少于10ms,接口平均时间少于30ms的短链接服务。
    \item 提供超过一万家网站的设别判断,为用户短链接提供返利处理及返利信息显示。
    \item 记录并分析短链接数据,为运营团队及用户提供分时点击量、设备信息等统计数据。
  \end{itemize}
  
  \item \textbf{DVM} {ShellScript} \hfill \href{github.com/ghosind/dvm}{github.com/ghosind/dvm}
  \begin{itemize}
    \itemsep -3pt {}
    \item ShellScript编写的Deno版本管理个人开源项目,提供了与NVM相似的版本管理功能。
    \item 项目在GitHub、Gitee等平台发布,并被列为Gitee官方推荐项目。
  \end{itemize}
  
\end{rSection}

\end{document}
