\documentclass{resume} % Use the custom resume.cls style

\usepackage{ctex}
\usepackage[left=0.4 in,top=0.4in,right=0.4 in,bottom=0.4in]{geometry}

\newcommand{\tab}[1]{\hspace{.2667\textwidth}\rlap{#1}}
\newcommand{\itab}[1]{\hspace{0em}\rlap{#1}}

\name{Chen Su}
\address{\href{mailto:ghosind@gmail.com}{ghosind@gmail.com} \\ \href{https://ghosind.com}{ghosind.com} \\ \href{github.com/ghosind}{github.com/ghosind}}

\begin{document}

%----------------------------------------------------------------------------------------
%----------------------------------------------------------------------------------------
%	EDUCATION SECTION
%----------------------------------------------------------------------------------------

\begin{rSection}{教育背景}

  {\bf 工学学士} 计算机科学与技术 \hfill {Jane 2017}

\end{rSection}

%----------------------------------------------------------------------------------------
% TECHINICAL STRENGTHS
%----------------------------------------------------------------------------------------
\begin{rSection}{个人技能}
  \begin{itemize}
    \itemsep -3pt {}
    \item 熟悉C、Golang、JavaScript/Typescript、Java以及C\#等编程语言,具有相应实际工作或项目经验。
    \item 具有国家化团队(美国、欧洲、东南亚)合作经验,有一定的英文阅读表达能力。
  \end{itemize}
\end{rSection}


%----------------------------------------------------------------------------------------
% WORK EXPERIENCE
%----------------------------------------------------------------------------------------
\begin{rSection}{工作经验}

\textbf{FatCoupon} \hfill 2019.5 - 至今\\
后端负责人 \hfill \textit{}
\begin{itemize}
  \itemsep -3pt {}
  \item 实现公司项目从0至1落地,并带领后端组实现日均万级DAU以及千万级接口调用服务支持。
  \item 主要负责服务架构体系设计,以及基于Golang、Node.js的后端服务设计与开发。
  \item 开发并维护包括Web服务、日志、ORM等在内的内部基础库。
  \item 优化数据缓存逻辑与网络拓扑结构,降低约85\%网络传输成本。
  \item 通过实现验证码、调用频率限制、设备/IP黑名单、接口熔断等机制,抵御单日百万级恶意登录行为。
\end{itemize}

\textbf{Timepop} \hfill 2018.8 - 2019.2\\
后端工程师 \hfill \textit{}
\begin{itemize}
  \itemsep -3pt {}
  \item 开发基于FoF算法的好友推荐系统,为用户提供伪实时二度好友推荐。
\end{itemize}

\textbf{上海店达} \hfill 2018.4 - 2018.8\\
后端工程师 \hfill \textit{}
\begin{itemize}
  \itemsep -3pt {}
  \item 
\end{itemize}

\textbf{杭州品茗} \hfill 2017.5 - 2017.11\\
软件工程师 \hfill \textit{}
\begin{itemize}
  \itemsep -3pt {}
  \item 
\end{itemize}

\end{rSection}


%----------------------------------------------------------------------------------------
% PROJECTS
%----------------------------------------------------------------------------------------
\begin{rSection}{项目经历}
  \vspace{-1.25em}
  
  \item \textbf{FatCoupon} {} \hfill \href{www.fatcoupon.com}{fatcoupon.com}
  \begin{itemize}
    \itemsep -3pt {}
    \item 设计并实现基于AWS EKS(kubernetes)、Docker、AWS Lambda函数计算构建微服务与无服务(serverless)架构体系,并通过HTTP、gRPC、消息队列以及Lambda调用实现服务间通信。
    \item 基于AWS服务以及消息队列构建消息推送、邮件等基础服务,实现订阅列表、模板渲染以及十万级消息的处理。
    \item 定时从上游服务获取并分析处理订单数据,实现每日十万级订单数据的检测、创建、更新以及其激励机制的处理。
  \end{itemize}
  
  \item \textbf{Kash.ly短链接系统} {Golang, MySQL, AWS} \hfill \href{kashly.net}{kashly.net}
  \begin{itemize}
    \itemsep -3pt {}
    \item 基于MD5、LCG以及Base62算法实现短链接服务,并加入校验位提供合法性校验。
    \item 初期提供约一千四百万短链接资源容量,并预留通过哈希算法以及数据库分表分库等方式,进行快速资源容量扩展能力。
    \item 记录并分析点击数据,提供历史点击量、点击设备等数据统计数据。
  \end{itemize}
  
  \item \textbf{DVM} {ShellScript} \hfill \href{github.com/ghosind/dvm}{github.com/ghosind/dvm}
  \begin{itemize}
    \itemsep -3pt {}
    \item 基于ShellScript实现的NVM-like Deno版本管理工具,并被列入Gitee官方推荐项目。
  \end{itemize}
  
\end{rSection}
  

\end{document}
