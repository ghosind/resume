\documentclass{resume} % Use the custom resume.cls style

\usepackage{ctex}
\usepackage[left=0.4 in,top=0.8in,right=0.4 in,bottom=0.4in]{geometry}

\newcommand{\tab}[1]{\hspace{.2667\textwidth}\rlap{#1}}
\newcommand{\itab}[1]{\hspace{0em}\rlap{#1}}

\name{Chen Su}
\address{\href{mailto:ghosind@gmail.com}{ghosind@gmail.com} \\ \href{https://ghosind.com}{ghosind.com} \\ \href{https://github.com/ghosind}{github.com/ghosind}}

\begin{document}

%----------------------------------------------------------------------------------------
%----------------------------------------------------------------------------------------
%	EDUCATION SECTION
%----------------------------------------------------------------------------------------

\begin{rSection}{教育背景}

  {\bf 工学学士} 计算机科学与技术 \hfill {2017.6}

\end{rSection}

%----------------------------------------------------------------------------------------
% TECHNICAL STRENGTHS
%----------------------------------------------------------------------------------------
\begin{rSection}{个人技能}
  \begin{itemize}
    \itemsep -3pt {}
    \item 熟悉C、Golang、JavaScript/TypeScript等编程语言,具有相应的的工作或项目经验。
    \item 具有分布式后端服务架构设计经验,有能力独立或参与团队进行架构设计与技术选型。
    \item 熟悉常用的MySQL、MongoDB、Redis等数据库系统,对ElasticSearch有一定了解。
    \item 对Linux系统有一定的了解,熟悉Linux下基本命令的使用以及Shell脚本的编写。
    \item 熟悉Docker使用及Dockerfile的编写,对Kubernetes有一定了解。
    \item 熟悉常用的HTTP、gRPC等通信方式,对GraphQL、SOAP等协议有所了解。
    \item 熟悉AWS云服务,具有四年基于AWS云服务的应用程序开发经验,对阿里云及微软Azure有一定了解。
    \item 了解消息队列系统机制,有RabbitMQ与ActiveMQ使用经验。
    \item 具有GitHub Actions、GitLab CI/CD等自动化工具使用经验,熟悉通过Shell脚本构建自动化工具。
    \item 具有国际化团队合作经验(北美、欧洲、东南亚),有一定的英文阅读表达能力。
    \item 对开源具有热情,具有多个开源项目的参与经历。
  \end{itemize}
\end{rSection}

%----------------------------------------------------------------------------------------
% WORK EXPERIENCE
%----------------------------------------------------------------------------------------
\begin{rSection}{工作经验}

\textbf{FatCoupon} \hfill 2019.5 - 至今\\
后端负责人/后端工程师 \hfill \textit{}
\begin{itemize}
  \itemsep -3pt {}
  \item 作为主要负责人参与了多个项目从0到1落地,并负责或参与后续的研发与维护。
  \item 负责服务端基于AWS Lambda、Docker等基础服务设计并构建跨项目架构体系,开发基于Golang、Node.js等语言实现的服务端系统。
  \item 实现并维护内部使用的Web服务、数据库处理、消息处理等基础框架,构建支付网关、动态配置中心、任务管道等基础设施服务,建立通用邮件、消息推送等跨项目中间件服务。
  \item 优化基于AWS云服务的内部网络拓扑结构,设计多级缓存逻辑,降低超75\%的内部网络通信成本。
  \item 构建基于设备设别、验证码校验等方式的服务安全体系,防御脚本恶意登录等攻击性行为。
  \item 开发内部运维自动化工具,提供服务健康检测汇报、冷数据归档、日志备份处理等功能。
  \item 基于React、AntDesign构建服务端控制平台,实现服务配置管理等基础服务的图形化处理界面。
\end{itemize}

\textbf{Timepop} \hfill 2018.8 - 2019.2\\
后端工程师 \hfill \textit{}
\begin{itemize}
  \itemsep -3pt {}
  \item 作为主要负责人开发并维护基于Node.js、MySQL、GraphQL的社交平台后端服务。
  \item 开发基于FoF算法的好友推荐系统,为用户提供伪实时二度好友推荐。
  \item 参与基于AWS云服务、Docker的服务端架构设计,基于GitLab CI/CD和Shell脚本实现服务自动化部署。
  \item 参与部分移动端React Native界面及基于Websocket的即时通信功能实现。
\end{itemize}

\textbf{上海店达} \hfill 2018.4 - 2018.8\\
后端工程师/小组Leader \hfill \textit{}
\begin{itemize}
  \itemsep -3pt {}
  \item 主要负责通过Node.js开发基于Express框架的后端服务,并编写前端页面及交互脚本。
  \item 作为采购服务组负责人对接产品经理、线下业务团队,确定需求内容及排期。
  \item 优化历史遗留问题,稳定化多个复杂业务查询逻辑,将多个频发超时接口稳定至常数时间内返回。
  \item 带领小组成员制定代码规范,对部分老服务重构优化。
  \item 参与内部系统前端页面的构建,实现合同打印等功能。
\end{itemize}

\textbf{杭州品茗} \hfill 2017.5 - 2017.11\\
软件工程师 \hfill \textit{}
\begin{itemize}
  \itemsep -3pt {}
  \item 基于Autodesk Revit、C\#、C++开发三维建筑信息管理系统。
  \item 开发基于WPF、WinForm以及MVVM设计模式的Windows桌面程序。
  \item 参与基于2D CAD图纸的3D模型建模功能开发,实现管道建模等功能。
  \item 参与跨团队间基于SQLite实现BIM项目与CAD的数据共享开发。
\end{itemize}

\end{rSection}

%----------------------------------------------------------------------------------------
% PROJECTS
%----------------------------------------------------------------------------------------
\begin{rSection}{项目经历}
  \vspace{-1.25em}

  \item \textbf{FatCoupon} {} \hfill \href{https://fatcoupon.com}{fatcoupon.com}
  \begin{itemize}
    \itemsep -3pt {}
    \item 设计并实现基于Docker、AWS Lambda函数计算构建微服务与Serverless相结合的架构体系,并通过HTTP、gRPC、消息队列以及Lambda调用实现服务间通信。
    \item 基于AWS服务以及消息队列构建消息推送、邮件、短信等基础服务,实现订阅列表、模板渲染等功能,在服务商频率限制内提供日均百万级消息处理能力。
    \item 设计并实现定时订单获取系统,从多个上游服务获取并分析处理订单数据,实现日均百万级订单数据的处理。
    \item 扩展AWS CloudFront CDN功能,提供自定义请求参数控制实现修改图片尺寸、缓存第三方图片等功能。
  \end{itemize}

  \item \textbf{FlashGiftcard} {} \hfill \href{https://flashgiftcard.com}{flashgiftcard.com}
  \begin{itemize}
    \itemsep -3pt {}
    \item 实现基于短信、邮件验证码的用户登录流程,通过加密证书版本管理及自动刷新机制提高用户鉴权安全性。
    \item 设计并实现可配置化多用户组全局权限管理系统,提供毫秒级权限验证。
    \item 构建支持国内外多种支付方式的支付网关,支持单支付方式多上游支付渠道的切换。
  \end{itemize}

  \item \textbf{Kash.ly返利短链接系统} {} \hfill \href{https://kashly.net}{kashly.net}
  \begin{itemize}
    \itemsep -3pt {}
    \item 设计基于LCG、Base62算法与MySQL实现的非线性伪随机短链接生成服务,并提供合法性检查减少错误链接带来的数据库请求资源消耗。
    \item 通过AWS Lambda部署短链接跳转服务,提供并发下自扩容能力,服务核心逻辑平均处理时间少于5ms,接口平均时间少于30ms。
    \item 提供超过一万家网站的识别判断,为用户短链接提供返利处理及返利信息显示。
    \item 记录并分析短链接数据,为运营团队及用户提供分时点击量、设备信息等统计数据。
  \end{itemize}

  \item \textbf{DVM} {} \hfill \href{https://github.com/ghosind/dvm}{github.com/ghosind/dvm}
  \begin{itemize}
    \itemsep -3pt {}
    \item ShellScript编写的Linux/MacOS平台下Deno版本管理开源项目,提供了Deno版本安装、卸载、版本间切换、别名管理以及版本信息错误检测等功能。
    \item 设计基于Git的自我版本管理,实现通过命令完成DVM的一键版本升级。
    \item 项目在GitHub、Gitee等平台发布,提供镜像选择功能,并被列为Gitee官方推荐项目。
  \end{itemize}

\end{rSection}

\end{document}
